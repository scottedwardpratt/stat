\section{Input Command File}

There is a front-end for \coral\, called \reefer.  \reefer\, takes a text file as 
input and attempts to carry out the commands listed in this file.  The commands
all have the following syntax:
\begin{verbatim}
<action> <input_string> <output_string> {
    [options]
}
\end{verbatim}
Note that the different tokens are separated by whitespace and any or all of a
line may be commented out with the comment string (by default this is set to
``\#'', but may be changed in the global preferences).  Here {\tt 
<input\_object>} and {\tt <output\_object>} are 
either {\tt Object}s, such as a correlations or a sources, or file names.  
Table~\ref{commands} is a complete list of commands and what things they act 
on.
\begin{table*}
\begin{tabular}{|c|c|c|l|}
    \hline\hline
    Command & First Argument  & Second Argument & Short Description\\ \hline\hline
    \multicolumn{4}{|l|}{\bf Input/Output Commands}\\ \hline
    {\tt read} & object name & file name  & Reads {\tt Object} from file \\ \hline
    {\tt write} & object name & file name & Writes {\tt Object} to file\\ \hline
    {\tt create} & object name & object type & Creates an {\tt Object}\\ \hline
    \multicolumn{4}{|l|}{\bf Correlation/Source Processing Commands}\\ \hline
    {\tt expand} & name of a {\tt corr\_3d\_cart} & name of a {\tt corr\_3d\_sphr} & Expands correlation in $Y_{\ell m}$'s\\ \hline
    {\tt unexpand} & name of a {\tt corr\_3d\_sphr} &  name of a {\tt corr\_3d\_cart}  & Converts a correlation from spherical to cartesian coords.\\
                   &                                &                                  & (inverse of {\tt expand})\\ \hline
    {\tt image} & name of a correlation & name of a source & Images a correlation\\ \hline
    {\tt unimage} & name of a source & name of a correlation & Reconstructs correlation from imaged source\\ 
                  &                  &                       & (inverse of {\tt image})\\ \hline
    \multicolumn{4}{|l|}{\bf Commands for Characterizing Correlation/Sources}\\ \hline
    {\tt gaussparam} & name of a source  & n/a & Does a cheezy ``fit'' to a Gaussian\\ \hline
    {\tt intbump} & object name & n/a & Integrates the volume under the bump of a corr or source\\ \hline
    {\tt powspec} & name of a {\tt corr\_3d\_sphr} & n/a & Computes power spectrum of source as function of $\ell$\\ \hline
    \multicolumn{4}{|l|}{\bf Plotting Commands}\\ \hline
    {\tt slicerad} & object name & file name & description\\ \hline
    {\tt slices} & object name & file name & description\\ \hline
    {\tt sliceo} & object name & file name & description\\ \hline
    {\tt slicel} & object name & file name & description\\ \hline
    {\tt sliceso} & object name & file name & description\\ \hline
    {\tt slicesl} & object name & file name & description\\ \hline
    {\tt sliceol} & object name & file name & description\\ \hline
    \multicolumn{4}{|l|}{\bf Misc. Commands}\\ \hline
    {\tt stop} & n/a & n/a & Stops script execution and exits program\\ \hline
    {\tt exit} & n/a & n/a & Alias for {\tt stop}\\ \hline
    {\tt quit} & n/a & n/a & Alias for {\tt stop}\\ \hline
    {\tt list} & n/a & n/a & Lists all objects in the Object Registry\\ \hline
    {\tt help} & n/a & n/a & Lists all of the {\tt reefer} commands\\ \hline
    {\tt help} & command name & n/a & Help for specific command\\ \hline
    {\tt delete} & object name & n/a & Deletes object from Object Registry\\ \hline
    {\tt setprefix} & object name  & new prefix & Renames the file prefix of all terms in a 3d object\\ \hline
    {\tt rename} & old object name & new object name  & Renames an object\\ \hline
    {\tt typeof} & object name & n/a  & Prints type of object\\ \hline
    {\tt print} & object name & n/a  & Prints object\\ \hline
    {\tt import} & file name & n/a & imports and processes {\tt file name} \\ \hline
    {\tt preferences} & n/a & n/a  & Prints global preferences\\ \hline
    \multicolumn{4}{|l|}{\bf Unimplemented Commands}\\ \hline
    {\tt copy} & object name 1 & object name 2 & Copies object 1 to 2\\ \hline
    {\tt copyterm} & object name 1 & object name 2 & Copies on term of object 1 to 2\\ \hline
    {\tt cd} & directory & n/a  & Changes working directory\\ \hline
    {\tt fit} & ?? & ?? & unimplemented\\ \hline
    {\tt fixtail} & name of a correlation & n/a & unimplemented\\ \hline
    {\tt chi2} & name of a correlation & name of a correlation & unimplemented\\ \hline
    {\tt boost} & object name & n/a & unimplemented\\ \hline
    {\tt } &  &  & description\\ \hline\hline
\end{tabular}
\caption{\reefer\, commands.}
\label{commands}
\end{table*}

\begin{table*}
\begin{tabular}{|c|c|c|c|}
    \hline\hline
    Object  & Short Description\\ \hline\hline
    \multicolumn{2}{|l|}{\bf Source Functions}\\ \hline
    {\tt source\_1d\_term}  &description\\ \hline
    {\tt source\_3d\_sphr} &  description\\ \hline
    {\tt source\_1d\_gaussian} &  description\\ \hline
    {\tt source\_1d\_2gaussian} &  description\\ \hline
    {\tt source\_1d\_blastwave} &  description\\ \hline
    {\tt source\_3d\_gaussian} &  description\\ \hline
    {\tt source\_3d\_blastwave} &  description\\ \hline
    {\tt source\_1d\_crab} &  description\\ \hline
    {\tt source\_3d\_crab} &  description\\ \hline
    \multicolumn{2}{|l|}{\bf Correlation Functions}\\ \hline
    {\tt corr\_1d\_term} &  description\\ \hline
    {\tt corr\_3d\_cart} &   description\\ \hline
    {\tt corr\_3d\_sphr} &   description\\ \hline
    \multicolumn{2}{|l|}{\bf Emission Functions}\\ \hline
    {\tt emissfunc\_blastwave}& description\\ \hline
    {\tt } &  description\\ \hline\hline
\end{tabular}
\caption{\reefer\, objects.}
\label{objects}
\end{table*}

When a new object is created, \coral\, attempts to create it with reasonable
defaults.  When a \reefer\, command is executed, \reefer\, uses these defaults
unless overidden in the input file.  To control the way a new object is made, 
add a section after a command as follows:
\begin{verbatim}
<action> <input_string> <output_string>{
    [option 1] = <value 1>
    [option 2] = <value 2>
    .
    .
    .
    <datablock> {
        <data in columns>
        .
        .
        .
        
    }
    <string_list> {
        <string>
        .
        .
        .
        
    }
}
\end{verbatim}

{\tt Object}s are also described in this manner, but with a few additions.  
The main difference being that {\tt Object} descriptions may have nested
sections:
\begin{verbatim}
<object> {
    [option 1] = <value 1>
    [option 2] = <value 2>
    .
    .
    .
    <datablock> {
        <data in columns>
        .
        .
        .
        
    }
    <string_list> {
        <string>
        .
        .
        .
        
    }
    .
    .
    .
}
\end{verbatim}
The purpose of such subsections is to store the actual data of the object.  They
are described in each {\tt object}'s description.
