% Outline for statistics tutorial
% Hal Canaray and Cory Quammen
%
% Note: this is in a rough format that may not compile in latex in its current form

\section{Statistics background}

\subsection{Basic concepts}

We have a model and we're trying to find parameters for it that best explain some observed phenomena.

\subsubsection{Probability and statistics review}

Touch on terms.

Probability, expectation, distribution, joint distribution, variance, covariance

\subsubsection{Joint implausbility}

\section{Gaussian process emulator}

\subsection{Background}

\begin{itemize}

See Chris Coleman-Smith's presentation at the June 2012 MADAI meeting on this

\item What it does?

\item Why use it instead of interpolation?

\item How to use it?

\end{itemize}

\subsection{Training an emulator}

\subsubsection{What is a nugget?}

\subsubsection{Inputs}

\begin{itemize}

\item Set regression order - this subtracts a broad trend in the data to make the training better behaved

\item Pick a covariance function - this is a kernel that determines the weights of nearby training data on an output

\item PCA variance (0.0 - 1.0) - how much of the variance do you want to explain?

\end{itemize}

\subsection{Verify emulator}

Feed training points back into emulator. Results won't be identical, but should be close. Jittering parameter-space position should give you similar values.

\section{Markov Chain Monte Carlo}

\subsection{Background}

\subsection{Algorithm we use}

Metropolis-Hastings

\subsection{Burn-in}

\subsection{Post burn-in}



\section{Software session}

\subsection{Download code}

\subsubsection{MADAI Workbench}

\subsubsection{Emulator}

Set up environment for running it

\subsubsection{Data}

\begin{itemize}

\item One die example data in format useful for training. Question: How long is this going to take?

\item Two die example

\end{itemize}

